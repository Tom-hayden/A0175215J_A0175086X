\section{Introduction}
    \pagestyle{tom}
    \sectionauthor{T. Hayden \& M. Gini}

Machine learning has been the object of intense research over the last few decades. It has produced tangible useful results over many different fields from beating the best players at Go (see section \ref{sec:Recent_trends}) to allowing users to automatically tag their friends in uploaded photo albums (see section \ref{sec:Application_img_rec}). Whilst machine learning has many pitfalls, due to its ability to be applied to any field where large amounts of data can be collected.

The object of this assignment is to give us some exposure into how machine learning can be applied to practical problems. This report will focus on how a multilayer perceptron can be trained on the CIFAR-10 data set to perform image recognition tasks. The CIFAR-10 data set contains 60000 $32\times32$ RGB images labeled into 10 classes. It is a popular benchmark used for image classification algorithms. It is useful because it is a relatively small data set and so training takes a relatively short amount of time.

In the following sections of the report, we give a brief overview of general applications of machine learning as well as some some notable recent accomplishments. The state-of-the-art algorithms
