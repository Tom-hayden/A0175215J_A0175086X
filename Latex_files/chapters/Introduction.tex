\section{Introduction}
    \pagestyle{tom}
    \sectionauthor{T. M. Hayden \& M. Gini}

Machine learning and especially artificial neural networks (ANN) have been the object of intense research over the last few decades. Tangible useful results have been obtained over many different fields, from achieving superhuman performance in the game Go (see Section \ref{sec:Recent_trends}) to allowing users of social networks to automatically tag their friends in photo albums (see Section \ref{sec:Application_img_rec}). Whilst there is a lack of the theoretical foundations behind ANN, it is still regarded as the state-of-the-art in many application areas.

The object of this assignment is to give us some exposure into how machine learning - especially ANN - can be applied to practical problems. This report will focus on how a multi-layer perceptron (MLP) can be trained on the CIFAR-10 dataset to perform image classification tasks. The CIFAR-10 data set contains 60000 $32\times32$ RGB images labeled into 10 classes. It is a popular benchmark for image classification algorithms. Due to its relatively small size, training takes a relatively short amount of time which makes it suited for our project.

In the following sections of the report, we give a brief overview of general applications of ANN as well as some some notable recent accomplishments. The state-of-the-art algorithms developed for the CIFAR-10 dataset are reviewed as well. Section \ref{sec:MLP} summarizes our work implementing a MLP and Section \ref{sec:CNN} presents the results of implementing a convolutional neural network (CNN) which is a more advanced network architecture.
