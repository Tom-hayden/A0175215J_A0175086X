\documentclass[12pt, a4paper]{article}

\usepackage[margin=20mm]{geometry}
\usepackage{helvet}
\renewcommand{\familydefault}{\sfdefault}
\usepackage{setspace}
\usepackage{url}

\usepackage{mathtools}
\usepackage{amssymb}
\usepackage{amsmath}
\usepackage{mathrsfs}
\usepackage{enumerate}
\usepackage{caption}
\usepackage{subcaption}
\usepackage{gensymb}
\usepackage{wrapfig}
\usepackage[square,super,comma]{natbib}
% \usepackage{url}
\usepackage{hyperref}
\usepackage{multicol}
\usepackage{multirow}
\usepackage{lipsum}

\usepackage{fancyhdr}
\setlength{\headheight}{13pt}
\fancypagestyle{tom}{
    \fancyhf{}
    \pagestyle{fancy}
    \cfoot{\thepage}
    \rhead{\rightmark}
    \lhead{T. Hayden}
}
\fancypagestyle{mario}{
    \fancyhf{}
    \pagestyle{fancy}
    \cfoot{\thepage}
    \rhead{\rightmark}
    \lhead{M. Gini}
}

\usepackage[binary-units]{siunitx}
\sisetup{%
  mode = math,
  detect-all,
  exponent-product = \cdot,
  number-unit-separator=\text{\,},
  math-rm=\mathsf,
  text-rm=\sffamily,
}

% \usepackage[utf8]{inputenc}
\usepackage[english]{babel}
% \usepackage[nottoc]{tocbibind} %Adds "References" to the table of contents

\usepackage{graphicx}
\graphicspath{{images/},{../images/}}
\usepackage{wrapfig}

\makeatletter
\newcommand{\sectionauthor}[1]{%
  {\parindent0pt\vspace*{-5pt}%
  \linespread{1.1}\large\scshape#1%
  \par\nobreak\vspace*{15pt}}
  \@afterheading%
}
\makeatother

\usepackage{titlesec}

\setcounter{secnumdepth}{4}

\titleformat{\paragraph}
{\normalfont\normalsize\bfseries}{\theparagraph}{1em}{}
\titlespacing*{\paragraph}
{0pt}{3.25ex plus 1ex minus .2ex}{1.5ex plus .2ex}

\usepackage{subfiles}

\usepackage{caption}
\usepackage{subcaption}
\usepackage{float}

\numberwithin{equation}{section}

\begin{document}
    \begin{titlepage}

    \begin{center}
    \vspace*{6cm}
    \centering
    \Huge
    { CIFAR-10 Image Recognition }
    \vspace{2cm}

    \Huge
    {EE4305 Introduction to Fuzzy/Neural Systems}
    \vspace{0.5cm}
    
    \Large
    {Mario Gini

    Thomas Hayden}
    \vfill
    \vspace{0.8cm}
    \end{center}
    \end{titlepage}

    \newpage
    \tableofcontents
\section{Introduction}
    \pagestyle{tom}
    \sectionauthor{T. Hayden}
    
The CIFAR-10 dataset contains 60000 images bla bla.\\
Objectives of this project are: bla bla\\
Structure of the report is as follows: bla bla\\

\section{Literature Review}
     \pagestyle{mario}
     \sectionauthor{M. Gini}
     
The literature review concentrates mainly on CNN networks. State-of-the-art architectures for CIFAR-10\\
- Advantages and disadvantages of the architectures.\\

-Application areas of the architectures\\

Advanced optimization methods to mention:\\
-Dropout\\
-Batch normalization\\

\section{MLP Classifier}
	    \pagestyle{mario}
	    \sectionauthor{M. Gini}
	
\subsection{Data Preprocessing and Augmentation}

\begin{itemize}
	\item Normalization
	
	The input data is normalized to lie within the range [0,1].
	
	\item Mean subtraction
	
	To further normalize the data, a the mean is subtracted on a per-pixel basis.
	
	\item Data augmentation
	
	Experience shows that a larger training data set increases network performance
\end{itemize}

\subsection{Network Structure}
	
\begin{itemize}
	\item Basic structure
	
	Since this is a classification problem, parts of the network structure are fixed. The last layer consists of 10 nodes and a in a "softmax" configuration. PICTURE of basic structure.
	
	\item Number of hidden layers/nodes
	
	Parameter search over 1-3 hidden layers, 1-500 neurons
	 	
	 	
\end{itemize}

	
\subsection{Optimization of Further Network Parameters}

\begin{itemize}
	\item Different learning rates
	
	\item Different optimization methods
\end{itemize}
	    
\section{CNN network}
	\pagestyle{mario}
	\sectionauthor{M. Gini}
		
\section{Conclusion}
     \pagestyle{mario}
     \sectionauthor{M. Gini}


    \newpage
    \bibliographystyle{unsrtnat}
    \bibliography{bibliography}
\end{document}
